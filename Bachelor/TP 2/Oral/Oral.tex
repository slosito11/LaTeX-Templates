% !TEX TS-program = pdflatex
% !TEX encoding = UTF-8 Unicode

% This file is a template using the "beamer" package to create slides for a talk or presentation
% - Giving a talk on some subject.
% - The talk is between 15min and 45min long.
% - Style is ornate.

% MODIFIED by Jonathan Kew, 2008-07-06
% The header comments and encoding in this file were modified for inclusion with TeXworks.
% The content is otherwise unchanged from the original distributed with the beamer package.

\documentclass{beamer}


% Copyright 2004 by Till Tantau <tantau@users.sourceforge.net>.
%
% In principle, this file can be redistributed and/or modified under
% the terms of the GNU Public License, version 2.
%
% However, this file is supposed to be a template to be modified
% for your own needs. For this reason, if you use this file as a
% template and not specifically distribute it as part of a another
% package/program, I grant the extra permission to freely copy and
% modify this file as you see fit and even to delete this copyright
% notice. 
\usepackage{xcolor}

\mode<presentation>
{
	\usetheme{Frankfurt}
	% or ...
	
	\definecolor{UBCblue}{HTML}{318063} % UBC Blue (primary)
	\definecolor{UBCgrey}{HTML}{1a2e2c} % UBC Grey (secondary)
	
	\setbeamercolor{palette primary}{bg=UBCblue,fg=white}
	\setbeamercolor{palette secondary}{bg=UBCgrey,fg=white}
	\setbeamercolor{palette tertiary}{bg=UBCgrey,fg=white}
	\setbeamercolor{palette quaternary}{bg=UBCgrey,fg=white}
	\setbeamercolor{structure}{fg=UBCblue} % itemize, enumerate, etc
	\setbeamercolor{section in toc}{fg=UBCgrey} % TOC sections
	
	% Override palette coloring with secondary
	\setbeamercolor{subsection in head/foot}{bg=UBCgrey,fg=white}
	
	
	\setbeamercovered{transparent}
	% or whatever (possibly just delete it)
}


\usepackage[english]{babel}
% or whatever

\usepackage[utf8]{inputenc}
% or whatever

\usepackage{times}
\usepackage[T1]{fontenc}
% Or whatever. Note that the encoding and the font should match. If T1
% does not look nice, try deleting the line with the fontenc.


%%% Stefano's packages:
\usepackage{datetime}
	\newdate{date}{13}{04}{2022}

\usepackage{subcaption}	
\usepackage{hyperref}
\usepackage{cleveref}	
\usepackage{animate}
	
%%%not Stefano's workaround
	
\let\chyperref\cref % Save the orginal command under a new name
\renewcommand{\cref}[1]{\hyperlink{#1}{\chyperref{#1}}} % Redefine the \cref command and explictely add the hyperlink. 
	
	
	
	
%%%	

\title% (optional, use only with long paper titles)
{Oral M4}

\subtitle
{Moment of inertia} % (optional)

\author[Losito Stefano] % (optional, use only with lots of authors)
{S.~Losito\inst{1}}% \and S.~Another\inst{2}}
% - Use the \inst{?} command only if the authors have different
%   affiliation.

\institute[University of Geneva] % (optional, but mostly needed)
{
	\inst{1}%
	Student BA. Physics\\
	University of Geneva}
%	\and
%	\inst{2}%
%	Department of Theoretical Philosophy\\
%	University of Elsewhere}
% - Use the \inst command only if there are several affiliations.
% - Keep it simple, no one is interested in your street address.

\date[Oral] % (optional)
{\displaydate{date} / Oral}

\subject{Talks}
% This is only inserted into the PDF information catalog. Can be left
% out. 

\graphicspath{{imagini/}}

\logo{\includegraphics[height=1cm]{logo}}

% If you have a file called "university-logo-filename.xxx", where xxx
% is a graphic format that can be processed by latex or pdflatex,
% resp., then you can add a logo as follows:

% \pgfdeclareimage[height=0.5cm]{university-logo}{university-logo-filename}
% \logo{\pgfuseimage{university-logo}}



% Delete this, if you do not want the table of contents to pop up at
% the beginning of each subsection:
\AtBeginSubsection[]
{
	\begin{frame}{Table of Contents}
		\tableofcontents[currentsection,currentsubsection]
	\end{frame}
}


% If you wish to uncover everything in a step-wise fashion, uncomment
% the following command: 

%\beamerdefaultoverlayspecification{<+->}


\begin{document}
	\sloppy
	\begin{frame}
		\titlepage
	\end{frame}
	
%	\begin{frame}{Outline}
%		\tableofcontents[pausesections]
%		% You might wish to add the option 
%	\end{frame}
	
	
	% Since this a solution template for a generic talk, very little can
	% be said about how it should be structured. However, the talk length
	% of between 15min and 45min and the theme suggest that you stick to
	% the following rules:  
	
	% - Exactly two or three sections (other than the summary).
	% - At *most* three subsections per section.
	% - Talk about 30s to 2min per frame. So there should be between about
	%   15 and 30 frames, all told.
	
	
		\begin{frame}{Introduction}
		The purpose of this experience is to study the moment of inertia, through periodical rotation.
	\end{frame}
	
%	\begin{frame}{Summary}
%		
%		\tableofcontents
%	\end{frame}
	
\section{Theory}



\appendix
	\begin{frame}{Conclusion \&\& Questions}
		\begin{itemize}
			\item Applications(G.R., Gyroscope,...)
		\end{itemize}
	\end{frame}

\begin{frame}{Error propagation}
	\begin{align*}
		&\resizebox{.9\hsize}{!}{$\Delta I=\sqrt{\bigg(\frac{\Delta I_c\cdot T^2}{T'^2-T^2}\bigg)^2+\bigg(\frac{-I_c\cdot T^2}{\big(T'^2-T^2\big)^2}\ 2\cdot T'\cdot\Delta T'\bigg)^2+\bigg(\frac{2\cdot I_c\cdot T\cdot \Delta T}{T'^2-T^2}+\frac{I_c\cdot T^2}{\big(T'^2-T^2\big)^2}\ 2\cdot T\cdot\Delta T\bigg)^2}$}\\
		&\resizebox{.8\hsize}{!}{$\Delta D=\sqrt{\bigg(\frac{8\pi^2\Delta I_c}{T'^2-T^2}\bigg)^2+ \bigg(\frac{-4\pi^2I_c}{\big(T'^2-T^2\big)^2}\ 2\cdot T'\cdot\Delta T'\bigg)^2+\frac{4\pi^2I_c}{\big(T'^2-T^2\big)^2}\ 2\cdot T\cdot\Delta T\bigg)^2}$}\\
		&\resizebox{.45\hsize}{!}{$\Delta I_{exp}=\sqrt{\bigg(\frac{2T\Delta T\cdot D}{4\pi^2}\bigg)^2+\bigg(\frac{T^2\cdot \Delta D}{4\pi^2}\bigg)^2}$}\\
		&\resizebox{.55\hsize}{!}{$\Delta I_{th}=\sqrt{\bigg(\frac{2}{5}\ \Delta M\cdot R^2\bigg)^2+\bigg(\frac{2}{5}\ M\cdot 2R\Delta R\bigg)^2}$}
	\end{align*}
\end{frame}
\end{document}



